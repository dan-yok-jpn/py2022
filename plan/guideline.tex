\documentclass{jsarticle} 
\usepackage[dvipdfmx]{hyperref} 
\usepackage{pxjahyper} 
\usepackage{longtable} 
\usepackage{booktabs} 
\providecommand{\tightlist}{\setlength{\itemsep}{0pt}\setlength{\parskip}{0pt}} 
\begin{document} 
\hypertarget{ux6cb3ux5dddux5b9aux671fux6a2aux65adux30c7ux30fcux30bf}{%
\section{河川定期横断データ}\label{ux6cb3ux5dddux5b9aux671fux6a2aux65adux30c7ux30fcux30bf}}

\hypertarget{ux6cb3ux5dddux5b9aux671fux7e26ux6a2aux65adux30c7ux30fcux30bfux4f5cux6210ux30acux30a4ux30c9ux30e9ux30a4ux30f3}{%
\subsection{河川定期縦横断データ作成ガイドライン}\label{ux6cb3ux5dddux5b9aux671fux7e26ux6a2aux65adux30c7ux30fcux30bfux4f5cux6210ux30acux30a4ux30c9ux30e9ux30a4ux30f3}}

河川定期縦横断測量作業は
「\href{https://www.mlit.go.jp/river/shishin_guideline/kasen/pdf/sokuryo_youryo.pdf}{河川定期縦横断測量業務実施要領・同解説}」に従って実施する。
成果は同要領と「\href{https://www.mlit.go.jp/river/shishin_guideline/kasen/gis/pdf_docs/juoudan/guideline0805.pdf}{河川定期縦横断データ作成ガイドライン}」に従って作成する。
なお、成果の一つである数値データのファイル形式は CSV
形式とする(ガイドライン 縦断:12 頁、横断:13 頁)。
また、「\href{http://www.cals-ed.go.jp/mg/wp-content/uploads/sokuryou50.pdf}{測量成果電子納品要領}」
に従って電子納品する。

\hypertarget{ux30d5ux30a1ux30a4ux30ebux547dux540dux898fux7d04}{%
\subsection{ファイル命名規約}\label{ux30d5ux30a1ux30a4ux30ebux547dux540dux898fux7d04}}

KASEN/WZYOU\_A/DATA/WZ***{A4}.csv

\begin{itemize}
\tightlist
\item
  WZ : 定期縦横断測量(河川測量)
\item
  *** : 測線(規定なし)
\item
  {A4} : 横断測量成果(数値データ)
\end{itemize}

\hypertarget{ux30c7ux30fcux30bfux30d5ux30a9ux30fcux30deux30c3ux30c8}{%
\subsection{データフォーマット}\label{ux30c7ux30fcux30bfux30d5ux30a9ux30fcux30deux30c3ux30c8}}

\hypertarget{ux69cbux9020ux7269ux306aux3057ux57faux790eux30c7ux30fcux30bfux306eux6cb3ux5dddux69cbux9020ux7269ux30d5ux30e9ux30b0ux304cux30bcux30ed}{%
\subsubsection{1.
構造物なし(基礎データの河川構造物フラグがゼロ)}\label{ux69cbux9020ux7269ux306aux3057ux57faux790eux30c7ux30fcux30bfux306eux6cb3ux5dddux69cbux9020ux7269ux30d5ux30e9ux30b0ux304cux30bcux30ed}}

\begin{longtable}[]{@{}rll@{}}
\toprule
行 & データ & 備考\tabularnewline
\midrule
\endhead
1 & \protect\hyperlink{ux57faux790eux30c7ux30fcux30bf}{基礎データ}
&\tabularnewline
2 & \protect\hyperlink{ux6e2cux70b9ux30c7ux30fcux30bf}{測点データ}
&\tabularnewline
: & 測点データ &\tabularnewline
1+\emph{n} & 測点データ & \emph{n}:測点数\tabularnewline
\bottomrule
\end{longtable}

\hypertarget{ux69cbux9020ux7269ux3042ux308aux57faux790eux30c7ux30fcux30bfux306eux6cb3ux5dddux69cbux9020ux7269ux30d5ux30e9ux30b0ux304cux30bcux30edux4ee5ux5916}{%
\subsubsection{2.
構造物あり(基礎データの河川構造物フラグがゼロ以外)}\label{ux69cbux9020ux7269ux3042ux308aux57faux790eux30c7ux30fcux30bfux306eux6cb3ux5dddux69cbux9020ux7269ux30d5ux30e9ux30b0ux304cux30bcux30edux4ee5ux5916}}

\begin{longtable}[]{@{}rll@{}}
\toprule
行 & データ & 備考\tabularnewline
\midrule
\endhead
1 & 基礎データ &\tabularnewline
2 & 測点データ &\tabularnewline
: & 測点データ &\tabularnewline
1+\emph{n} & 測点データ & \emph{n}:測点数\tabularnewline
2+\emph{n} &
\protect\hyperlink{ux69cbux9020ux7269ux30c7ux30fcux30bf}{構造物データ}
&\tabularnewline
\bottomrule
\end{longtable}

\hypertarget{ux57faux790eux30c7ux30fcux30bf}{%
\subsection{基礎データ}\label{ux57faux790eux30c7ux30fcux30bf}}

\begin{longtable}[]{@{}rll@{}}
\toprule
カラム & 属性 & 備考\tabularnewline
\midrule
\endhead
1 & 距離標 &\tabularnewline
2 & 流心間距離 & 直下の測線までの距離(m)\tabularnewline
3 & 左岸距離杭高 &\tabularnewline
4 & 右岸距離杭高 &\tabularnewline
5 & 左岸水際杭高 &\tabularnewline
6 & 右岸水際杭高 &\tabularnewline
7 & 測点数 &\tabularnewline
8 & 左岸堤防法尻高 &\tabularnewline
9 & 左岸堤内地平均 &\tabularnewline
10 & 右岸堤防法尻高 &\tabularnewline
11 & 右岸堤内地平均 &\tabularnewline
12 & {河川構造フラグ} &
0:構造物なし、1:橋梁、2:堰、3:落差工、4:潜水橋\tabularnewline
13 & 測量年月日 & yyyymmdd\tabularnewline
14 & 河川番号 &
\href{http://www.thr.mlit.go.jp/bumon/b00037/k00290/river-hp/kasen/code/data/050401_siyou01\%5B1\%5D.pdf}{河川コード仕様書}\tabularnewline
15 & 水系名 & シフトJIS\tabularnewline
16 & 河川名称 & シフトJIS\tabularnewline
\bottomrule
\end{longtable}

\hypertarget{ux6e2cux70b9ux30c7ux30fcux30bf}{%
\subsection{測点データ}\label{ux6e2cux70b9ux30c7ux30fcux30bf}}

\begin{longtable}[]{@{}rll@{}}
\toprule
カラム & 属性 & 備考\tabularnewline
\midrule
\endhead
1 & 測点のタイプ & 下表\tabularnewline
2 & 距離 &\tabularnewline
3 & 高さ &\tabularnewline
\bottomrule
\end{longtable}

\hypertarget{ux6e2cux70b9ux306eux30bfux30a4ux30d7}{%
\subsubsection{測点のタイプ}\label{ux6e2cux70b9ux306eux30bfux30a4ux30d7}}

\begin{longtable}[]{@{}rll@{}}
\toprule
タイプ & 内容 & 備考\tabularnewline
\midrule
\endhead
0 & なし &\tabularnewline
1 & 左岸距離杭高 &\tabularnewline
2 & 距離杭下 &\tabularnewline
3 & 堤防法肩 &\tabularnewline
4 & 道路 & 始点・終点\tabularnewline
5 & 水路 & 始点・終点\tabularnewline
6 & 河川管理境界 &\tabularnewline
7 & 官民境界 &\tabularnewline
8 & 護岸(コンクリート) & 始点・終点\tabularnewline
9 & 護岸(石張) & 始点・終点\tabularnewline
10 & 護岸(コンクリートブロック) & 始点・終点\tabularnewline
11 & 護岸(蛇篭) & 始点・終点\tabularnewline
12 & 水際杭高 & 左岸・右岸\tabularnewline
13 & 低水路肩 & 左岸・右岸\tabularnewline
14 & アスファルト & 始点・終点\tabularnewline
15 & 水制 & 始点・終点\tabularnewline
16 & 水面 & 左岸・右岸\tabularnewline
17 & 根固ブロック & 始点・終点\tabularnewline
18 & 右岸距離杭高 &\tabularnewline
19 & 鉄柵 & 地盤・上端\tabularnewline
20 & 生垣 & 地盤・上端\tabularnewline
21 & ブロック塀 & 始点・終点\tabularnewline
22 & ガードレール & 地盤・上端\tabularnewline
23 & 家屋 & 地盤\tabularnewline
24 & 左岸本堤法尻 & 川表・川裏\tabularnewline
25 & 右岸本堤法尻 & 川表・川裏\tabularnewline
26 & その他堤防法尻 &\tabularnewline
27 & 看板 & 始点・終点\tabularnewline
28 & 堤内地 & 始点・終点\tabularnewline
\bottomrule
\end{longtable}

\hypertarget{ux69cbux9020ux7269ux30c7ux30fcux30bf}{%
\subsection{構造物データ}\label{ux69cbux9020ux7269ux30c7ux30fcux30bf}}

\hypertarget{ux6a4bux811a}{%
\subsubsection{橋脚}\label{ux6a4bux811a}}

\begin{longtable}[]{@{}rll@{}}
\toprule
カラム & 属性 & 備考\tabularnewline
\midrule
\endhead
1 & 構造物名 & シフトJIS\tabularnewline
2 & 橋脚の投影幅 &\tabularnewline
3 & 橋脚の本数 &\tabularnewline
\bottomrule
\end{longtable}

\hypertarget{ux5830}{%
\subsubsection{堰}\label{ux5830}}

\begin{longtable}[]{@{}rll@{}}
\toprule
カラム & 属性 & 備考\tabularnewline
\midrule
\endhead
1 & 堰高 &\tabularnewline
2 & 堰天端高 &\tabularnewline
3 & 堰幅 &\tabularnewline
4 & 堰上流側勾配 &\tabularnewline
5 & 堰下流側勾配 &\tabularnewline
\bottomrule
\end{longtable}

\hypertarget{ux843dux5deeux5de5}{%
\subsubsection{落差工}\label{ux843dux5deeux5de5}}

\begin{longtable}[]{@{}rll@{}}
\toprule
カラム & 属性 & 備考\tabularnewline
\midrule
\endhead
1 & 落差 &\tabularnewline
2 & 幅 &\tabularnewline
\bottomrule
\end{longtable}

\hypertarget{ux6f5cux6c34ux6a4b}{%
\subsubsection{潜水橋}\label{ux6f5cux6c34ux6a4b}}

\begin{longtable}[]{@{}rll@{}}
\toprule
カラム & 属性 & 備考\tabularnewline
\midrule
\endhead
1 & 橋脚の投影幅 &\tabularnewline
2 & 橋脚の長さ &\tabularnewline
3 & 桁の厚さ &\tabularnewline
4 & 桁の長さ &\tabularnewline
5 & 橋脚の本数 &\tabularnewline
\bottomrule
\end{longtable}
\end{document} 
